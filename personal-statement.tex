I have worked in schools for 17 years, both in leadership and governance roles. In September 2021 I decided to take a career break from full-time work in schools and focus working as a consultant to support schools in continuous improvement. I am passionate about governance and the importance of supporting governors from diverse backgrounds to understand their role and drive school improvement. Having recently been elected as Chair of Governors, working with the NGA would give me the opportunity to both use my experience to support schools and develop my own understanding of governance to the next level.

I have been on the governing body of Christ's School in Richmond since 2017, serving on both the Teaching \& Learning Committee and Finance, Premises \& Staffing Committees. I have recently been elected Chair of Governors and feel immensely privileged to be taking on this role. During my time at Christ's I have consistently supported and pushed the school, particularly throughout the COVID-19 pandemic where the work of the education committee (and other governors) helped support the school to implement excellent online learning provision early on. 

In taking over as Chair of the governing body I am focussed on building the capacity of the governing body to work as a team and draw on all the knowledge and skills represented. I am an experienced chair of meetings and have the support of an excellent clerk but I am alive to the challenges faced by all voluntary members of governing bodies at this time. While the experience of the pandemic has super-charged online ways of working (something I support), this has led to new challenges in team building and ensuring strong relationships between governors, senior leadership and other stakeholders. If I am successful in this application then I would be constantly evaluating the impact of new ways of working on the effectiveness of teamwork for governing bodies.

In my time at Christ's School I have been through all aspects of the function of a governing body, including a full governance review, an Ofsted inspection, regular self-evaluation and a process of strategy development to create a long-term vision for the school. As a consultant for the NGA I would be able to draw on this knowledge and experience in supporting schools in implementing the eight elements of effective governance in their settings.

As a senior leader at Westminster School I worked with the governing body through the education committee, giving me an opportunity to gain experience of the work of governing bodies from a school's point of view. During this time I worked with the governors on a number of strategic projects, in particular making a significant, material change to the school's admissions processes. This involved a significant range of stakeholders including parents and feeder schools as well as ensuring all changes were fully compliant with our own policies and DfE regulations. I witnessed the benefits an effective governing body can bring to strategic change, as well as the importance of a shared vision between governors and senior leaders. At Westminster I was also involved in working closely with Harris Westminster Sixth Form and experienced the benefits that being part of a MAT brings to schools at a strategic level, as well as the importance of schools balancing their own individual identities with that of the MAT. 

It is essential that governing bodies are aware of their responsibilities and are compliant with all relevant regulations. In my time as an inspector for ISI I have learnt how to check schools' compliance with regulations and best practice in a way which avoids conflict and builds trustful and supportive relationships. I bring this understanding to my work as a consultant for schools on regulatory compliance where I have developed skills in evaluating policies and practices, but more importantly feeding back to clients with clarity. My experience delivering INSET in schools, both in my previous full-time roles and as a consultant, has also helped me develop the skills needed to lead change effectively. For example, at Westminster I led a complete change in the approach to supporting students with SEND and as a trainer have advised schools in moving away from superficial interventions to more embedded change.

Underlying all my work is my personal committent to integrity and leadership, and a focus on the importance of student-outcomes which provides the motivation for undertaking courageous conversations. One time when this was particularly tested was when I was asked to conduct an investigation into a grievance at another school. This was as a result of a particularly tricky stage 3 complaint involving alleged racial discrimination against a child with SEND. As part of the investigation I found that there had been mistakes in the original complaint investigation. Rectifying this involved difficult conversations including suggesting the school had to review their reports to the LADO and other parties. By being extremely clear and fair in my discussions, and not shying away from challenging issues, I was able to support those involved in achieving a just resolution to the situation.

In summary, I feel that in my experience and approach I am well suited to supporting governing bodies on behalf of the NGA. I am committed to my own personal development and am excited about the opportunity to work with the NGA in this role.